% mnras_template.tex
%
% LaTeX template for creating an MNRAS paper
%
% v3.0 released 14 May 2015
% (version numbers match those of mnras.cls)
%
% Copyright (C) Royal Astronomical Society 2015
% Authors:
% Keith T. Smith (Royal Astronomical Society)

% Change log
%
% v3.0 May 2015
%    Renamed to match the new package name
%    Version number matches mnras.cls
%    A few minor tweaks to wording
% v1.0 September 2013
%    Beta testing only - never publicly released
%    First version: a simple (ish) template for creating an MNRAS paper

%%%%%%%%%%%%%%%%%%%%%%%%%%%%%%%%%%%%%%%%%%%%%%%%%%
% Basic setup. Most papers should leave these options alone.
\documentclass[a4paper,fleqn,usenatbib]{mnras}

% MNRAS is set in Times font. If you don't have this installed (most LaTeX
% installations will be fine) or prefer the old Computer Modern fonts, comment
% out the following line
\usepackage{newtxtext,newtxmath}
% Depending on your LaTeX fonts installation, you might get better results with one of these:
%\usepackage{mathptmx}
%\usepackage{txfonts}

% Use vector fonts, so it zooms properly in on-screen viewing software
% Don't change these lines unless you know what you are doing
\usepackage[T1]{fontenc}
\usepackage{ae,aecompl}

%%%%% AUTHORS - PLACE YOUR OWN PACKAGES HERE %%%%%

% Only include extra packages if you really need them. Common packages are:
\usepackage{graphicx}	% Including figure files
\usepackage{amsmath}	% Advanced maths commands
\usepackage{amssymb}	% Extra maths symbols
\usepackage[xindy]{glossaries}
\glsdisablehyper
\usepackage{subcaption}
\usepackage{tabularx}
\captionsetup{compatibility=false}
\usepackage[normalem]{ulem}

%%%%%%%%%%%%%%%%%%%%%%%%%%%%%%%%%%%%%%%%%%%%%%%%%%

%%%%% AUTHORS - PLACE YOUR OWN COMMANDS HERE %%%%%

% Please keep new commands to a minimum, and use \newcommand not \def to avoid
% overwriting existing commands. Example:
%\newcommand{\pcm}{\,cm$^{-2}$}	% per cm-squared

\newcommand{\GSF}[1]{\noindent\textcolor{blue}{GSF:#1}}
%comments by Marisa
\newcommand{\cM}[1]{\textcolor{magenta}{ #1 --M}}

%glossary
\newacronym{alfa}{ALFA}{Arecibo L-Band Feed Array}
\newacronym{dm}{DM}{Dispersion Measure}
\newacronym{frb}{FRB}{Fast Radio Burst}
\newacronym{fwhm}{FWHM}{Full-Width at Half-Maximum}
\newacronym{gbt}{GBT}{Greenbank Telescope}
\newacronym{if}{IF}{Intermediate Frequency}
\newacronym{igm}{IGM}{Intergalactic Medium}
\newacronym{ism}{ISM}{Interstellar Medium}
\newacronym{lo}{LO}{Local Oscillator}
\newacronym{nip}{NIP}{Non-image Processing}
\newacronym{pll}{PLL}{Phased-locked Loop}
\newacronym{rfi}{RFI}{Radio-frequency Interference}
\newacronym{ska}{SKA}{Square Kilometre Array}
\newacronym{sefd}{SEFD}{System Equivalent Flux Density}
\newacronym{snr}{S/N}{Signal-to-Noise Ratio}
\newacronym{sps}{SPS}{Single Pulse Search}
\newacronym{vlbi}{VLBI}{Very-Long Baseline Interferometry}
\newacronym{xao}{XAO}{Xinjiang Astronomical Observatory}

%%%%%%%%%%%%%%%%%%%%%%%%%%%%%%%%%%%%%%%%%%%%%%%%%%

% Title of the paper, and the short title which is used in the headers.
% Keep the title short and informative.
\title[FRB Detections and Verification Tests]{Reporting FRB Detections
and Verification Tests}

% Aris Karastergiou
% KJ Lee
\author[G. Foster et al.]{
Griffin Foster$^{1,2}$\thanks{E-mail: griffin.foster@physics.ox.ac.uk},
Marisa Geyer$^{3}$,
Mayuresh Surnis$^{4,5}$
\\
% List of institutions
$^{1}$University of Oxford, Sub-Department of Astrophysics, Denys Wilkinson Building, Keble Road, Oxford, OX1 3RH, United Kingdom\\
$^{2}$Department of Astronomy, University of California, Berkeley, 501 Campbell
Hall \#3411, Berkeley, CA, 94720, USA\\
$^{3}$SKA-SA, 3rd Floor, The Park, Park Road, Pinelands, South Africa\\
$^{4}$Department of Physics and Astronomy, West Virginia University, Morgantown, WV 26505, USA\\
$^{5}$Center for Gravitational Waves and Cosmology, West Virginia University, Chestnut Ridge Research Building, Morgantown,\\ WV 26505, USA\\
}

% These dates will be filled out by the publisher
\date{Accepted XXX. Received YYY; in original ZZZ}

% Enter the current year, for the copyright statements etc.
\pubyear{2018}

% Don't change these lines
\begin{document}
\label{firstpage}
\pagerange{\pageref{firstpage}--\pageref{lastpage}}
\maketitle

% TODO: The Great Restructure
% 1. Introduction
% 2. False-Positive FRB Detections
%   a. ALFA non-linear response
%   b. ALFA low-SNR
%   c. LOFAR RADAR
%   d. XAO RADAR
% 3. Verification Criteria
%   a. prototypical FRBs
%   b. detailed list of criteria
%   c. table of critera for FRBs, false-positives, pulsars
%   d. cases and odd FRBs
%   e. analysis of previous detections and criteria, reporting time scales
% 4. Detection Reporting and Future Observational Methods

% TODO: test software, run on previous frbs
% TODO: sent to simon, jon, leon oostrum (apertif confidence)

% Abstract of the paper
\begin{abstract}
% What is the point of the paper?
% What is the context of the study? What background information is necessary to understand the study?
% How was the study done?
% What is the main take away message?
% What can be said about these results, and how does this affect future work?
The one-off nature of most Fast Radio Bursts (FRBs) requires extra scrutiny when
reporting such detections as astrophysical.  The prototypical FRB is a broadband
signal, appearing to occur over a frequency range wider than the receiver
bandwidth, narrow-in-time, and highly dispersed, following a $\nu^{-2}$
relation.  But, some FRBs appear band-limited, show apparent scintillation,
complex frequency-dependent structure, or multi-component pulse shapes.  In a
search for rare signals in a noisy data set, the number of false-positive
detections is based on the detection threshold and signal filters.  Such
searches should find a number of false-positive events.  We present examples of
false-positive events that occur in multiple searches, which on initial
inspection appear to be FRB-like, but are found to be due to instrumental
variations, noise, and radio frequency interference (RFI).  Differentiating
these false-positive detections from astrophysical events, requires knowledge
and tests beyond performing a thresholded single-pulse detection.  We discuss
post-detection analysis, verification tests, and data sets which should be
provided when reporting an FRB detection.
\end{abstract}

% Select between one and six entries from the list of approved keywords.
% Don't make up new ones.
\begin{keywords}
radio continuum: transients -- methods: observational
\end{keywords}

%%%%%%%%%%%%%%%%% BODY OF PAPER %%%%%%%%%%%%%%%%%%

\section{Introduction}
\label{sec:intro}

The origin of \glspl{frb} continues to be a mystery since they were first
reported \citep{2007Sci...318..777L}. The \gls{dm} associated with the reported
events indicate they are occur well beyond our galaxy, possibly at cosmological
distances. They appear to be extremely bright, short duration events yet their
emission mechanism is not known.  This consensus has developed from the
reporting of detections with multiple telescopes, at different frequencies,
using different instrumentation. \glspl{frb} are difficult to detect as they
require high-gain telescopes which typically have a small beam size, such that
despite many thousands of observing hours only, a few dozen have been reported
as of this writing \citep{2016PASA...33...45P}.

The prototypical \gls{frb} is broad-band, appearing to be broader than most
receivers used in surveys. The pulse is narrow-in-time, on the order of a few
milliseconds in width, with a single component structure. The pulse is highly
dispersed, following a $\nu^{-2}$ relation and appears to be a one-off event
(only FRB121102 is known to repeat \citep{2016Natur.531..202S}).  Though, not
all reported detections appear to follow this prototypical form. Many show
complex frequency-dependent structure and are possibly band limited. The pulse
width varies due to either the emission process or propagation effects such as
scattering from the \gls{ism} and \gls{igm}.

These rare events are detected by automated GPU-accelerated software pipelines
that extensively search a broad range of \gls{dm} trials, pulse widths, and
starting times.  The de-dispersed time series is then thresholded, any
peaks above a minimum \gls{snr} are reported as potential detections. The number
of potential detections is usually overwhelming due to \gls{rfi} and system gain
variations. Initially, potential detections were reviewed manually. But, with
the amount of data acquired in recent surveys, it has become a significant time
effort to do so. Further, our understanding of the expected signal properties
has allowed the development of filters and models to select and prioritize
individual events \citep{2018MNRAS.474.3847F}.

As these sources are rare and generally appear not to repeat (except FRB121102),
there is an issue of verifiability. There is significant \gls{rfi} detectable at
all radio observatories, and there are known anthropogenic sources that appear
FRB-like \citep{2011ApJ...727...18B}.  Given the significant number of \gls{dm}
trials and high-time resolution of the spectra in a typical survey there are a
large number of false-positives which pass the automated post-processing
detection tests.  This is by design, as we would like to severely limit the
potential for false-negatives (type-II errors) in our detection pipelines by
accepting a number of false-positives (type-I errors).  But, given the large
sampling of the parameter search space it can be difficult, if not impossible to
differentiate between a true, astrophysical \gls{frb} (true-positive) and a
`terrestrial' \gls{frb} (false-positive) due to \gls{rfi}, systematics, or other
local effects. As the survey time increase the likelihood of detecting such a
false-positive will increase. To counter false detection of `terrestrial'
\glspl{frb}, verification criteria can be tested when reporting a detection.

In this paper, we present other examples of FRB-like sources which, after
further investigation, we show to be non-astrophysical. Using these and
previously reported \glspl{frb}, we develop a set of criteria to test detections
against and discuss the data, in addition to the detected dynamic spectrum, that
can be reported to provide a robust statement about an \gls{frb} detection.

\section{False-Positive FRB Detections}
\label{sec:false-pos}

During an \gls{frb} search survey there will be a significant number of
false-positive detections. The rate of these detections is set by the minimum
\gls{snr} threshold, the parameter search space, and the terrestrial environment
of the observatory. We use the term `terrestrial' to encompass multiple effects:
anthropogenic radio signal, variations in the observing system, and natural
signals from within the solar system.

Most false-positive events are filtered out with filters and classifier models.
The remaining events are commonly examined by eye as expert human knowledge
appears to be the best way to classify an detection but limitations in time do
not allow for all events to classified this way. On inspection, true
astrophysical events and terrestrial events can be difficult to differentiate
even by the expert.  These terrestrial events, indeed, should appear like
astrophysical \glspl{frb} as they have passed the detection pipeline tests. By
further investigation of the telescope state it is often possible to determine
the origin of an event. Though not always, this leads to the precarious nature
of reporting an \gls{frb} as astrophysical.  In this section we present examples
of such events which on initial inspection appear to be astrophysical, but after
further investigation prove to be terrestrial in origin.

%%%%%%%%%%%%%%%%%%%%%%%%%%%%%%%%%%%%%%%%%%%%%%%%%%%%%%%%%%%%%%%%%%%%%%%%%%%
\subsection{ALFA Gain Response}
\label{sec:D20161204}

% TODO: shorten this section
% TODO: seems a bit odd for ALFABURST to feature so prominently but no be in the abstract
In the two years of the initial ALFABURST survey \citep{2017ApJS..228...21C,
2018MNRAS.474.3847F}, over 200k 8.4-second data windows have been recorded in
which the \gls{frb} search pipeline detected an event above the minimum
\gls{snr} detection threshold of 10.  The vast
majority of these events were due to \gls{rfi} and instrumental variations,
while others were due to bright single pulses of known pulsars. Most of these
events were corretly classified with an automated classifier model. A small number of
windows contain FRB-like events which only on further inspection of the data,
the telescope operation status, and contextual information revealed, are from
local sources.

A narrow-in-time, broad-in-frequency, millisecond pulse was detected with the
ALFABURST system at MJD 57726.563263913 / Unix time 1480858266 (09:31:06 Arecibo
local time) in Beam 0 (the central beam) of the \gls{alfa}
receiver\footnote{http://www.naic.edu/alfa/} (Figure
\ref{fig:beam0_dynamic_spec_wide}). ALFABURST was processing 56~MHz of bandwidth
between 1457~MHz and 1513~MHz. The \gls{snr} of this pulse is maximized (10.46)
when the pulse is dedispersed with a \gls{dm} of 293~pc~cm$^{-3}$ and the
256~$\mu$s resolution is decimated by a factor of 16 to a time resolution of
4~ms. The dedispered time series shows an approximately 20~ms \gls{fwhm} pulse.
The dip before and after the pulse is due to the zero-DM filtering (i.e. the
moving average is subtracted) during pulse detection
\citep{2009MNRAS.395..410E}. This is a simple way to remove a drifting gain
baseline at the cost of removing some of the overall pulse power, particularly
at low DM trials. The bright, narrow-band signal at approximately $t=0.1$
seconds is locally generated \gls{rfi} which we cover later in this section.

\begin{figure*}
    % watermark:terrestrial-frb-letter/notebooks/ALFABURST_events.ipynb
    \includegraphics[width=1.0\linewidth]{figures/D20161204_buf23_Beam0_wide.pdf}
    \caption{Detected FRB-like event in Beam 0 of ALFA. The characteristic dip
    before and after the event is due to zero-DM removal which is part of the
    ALFABURST RFI exciser. The strong, narrowband, repeating signal is due to a
    local RFI source.
    }
    \label{fig:beam0_dynamic_spec_wide}
\end{figure*}

\begin{figure*}
    \centering
    % watermark:terrestrial-frb-letter/notebooks/ALFABURST_events.ipynb
    \begin{subfigure}[t]{0.45\textwidth}
        \centering\captionsetup{width=.95\linewidth}
        \includegraphics[width=1.0\textwidth]{figures/D20161204_buf23_Beam0.pdf}
        \caption{Detected FRB-like event in beam 0 of ALFA. The characteristic
        dip before and after the event is due to zero-DM removal which is part
        of the ALFABURST RFI exciser. The strong, narrowband source at 1480 MHz
        around 0.1~s is due to a local RFI source.}
        \label{fig:beam0_dynamic_spec}
    \end{subfigure}
    % watermark:terrestrial-frb-letter/notebooks/ALFABURST_events.ipynb
    \begin{subfigure}[t]{0.45\textwidth}
        \centering\captionsetup{width=.95\linewidth}
        \includegraphics[width=1.0\textwidth]{figures/D20161204_buf4_Beam5.pdf}
        \caption{Detected FRB-like event in beam 5 of ALFA. The event width
        appears wider than the Beam 0 event as the zero-DM dips are not as
        prominent.
        }
        \label{fig:beam5_dynamic_spec}
    \end{subfigure}
    \caption{
    Dynamic spectrum (top) and dedispersed time series (bottom) of an FRB-like
    event that was detected simultaneously in Beam 0 and 5 of the ALFA receiver
    on December 4, 2016. The dynamic spectrum has been bandpass normalized.
    }
    \label{fig:dynamic_spec}
\end{figure*}

On initial inspection, this event looks like a promising new astrophysical
\gls{frb}. The flux density of the event can be computed with the radiometer
equation
%
$$
S = \textrm{SEFD} \frac{\textrm{S/N}}{\sqrt{D \; \Delta \tau \;
\Delta \nu}},
$$
%
using a \gls{sefd} of 3~Jy for the \gls{alfa} receiver. This results in a flux
density of $S = 66$~mJy from Beam 0, which would be a lower flux than any
previously detected \gls{frb} \citep{2016PASA...33...45P}. This flux estimate is
a lower limit, as we are assuming the source is at the centre of the beam. The
width is large for \glspl{frb} but within the range of those previously
reported.

We inspected all other events in the same time window as the Beam 0 event. An
event was found in Beam 5 only (Figure \ref{fig:beam5_dynamic_spec}). This pulse
lines up with the Beam 0 event in time exactly, but the \gls{frb} was maximized
($S/N=16$) for a DM of 829~pc~cm$^{-3}$. Upon further inspection and testing
different DMs we found that this event appeared to decrease in width for lower
DM trials. There was \gls{rfi} clipping in this event which is known to
introduce a bias, resulting in a maximized \gls{snr} at a different DM trial.
The Beam 0 and Beam 5 event are the same event.

The Beam 5 detection has a lower \gls{snr} than the Beam 0 detection at a DM
trial of 293~pc~cm$^{-3}$. This is reasonable since the Beam 0 side-lobes
overlap with all the other beams, and the Beam 5 side-lobes overlap with the
Beam 0 primary beam. This would indicate that the sky source is somewhere
between the Beam 0 and Beam 5 pointing centres. And, the detection was from the
edge of the primary lobe or in the side-lobe of each beam.

The width of the dedispersed pulse in Beam 5 appears wider, but this is likely
due to the lower \gls{snr} of the event having a smaller effect on the spectrum
normalization.

Looking at the immediate period before and after the pulse we see that there are
no similar events (Figure \ref{fig:dm_time}). The event appears to be isolated
in time, with a fairly compact representation in DM-space (Figure
\ref{fig:dm_time_event}) similar to that of a single pulse from a high DM pulsar
such as PSR B1859+03 (Figure \ref{fig:dm_time_B1859}). The event would be
detected with significant \gls{snr} at higher \gls{dm} trials due to the large
width of the pulse, but peaks at a \gls{dm} trial of 293~pc~cm$^{-3}$.

\begin{figure}
    \centering
    % watermark:terrestrial-frb-letter/notebooks/ALFABURST_events.ipynb
    \begin{subfigure}[t]{0.5\textwidth}
        \centering\captionsetup{width=.95\linewidth}
        \includegraphics[width=1.0\textwidth]{figures/D20161204_dmtrials_buf23_Beam0.pdf}
        \caption{DM vs. time plot of for a 1.5~second window centred on the
        December 4th event in beam 0. The \gls{snr} peaks at a DM of
        293~pc~cm$^{-3}$. There is a significant detection at larger DM trials
        due to the width of the pulse.
        }
        \label{fig:dm_time_event}
    \end{subfigure}
    % watermark:terrestrial-frb-letter/notebooks/ALFABURST_events.ipynb
    \begin{subfigure}[t]{0.5\textwidth}
        \centering\captionsetup{width=.95\linewidth}
        \includegraphics[width=1.0\textwidth]{figures/B1859_dmtrials.pdf}
        \caption{DM vs. time plot of a bright single pulse from B1859+03 which
        has a DM of 402~pc~cm$^{-3}$ and pulse width of 11~ms (W50).
        }
        \label{fig:dm_time_B1859}
    \end{subfigure}
    \caption{DM-space plot shows the characteristic butterfly pattern of the
    narrow-in-time, dispersed pulse detected by ALFABURST. A single pulse
    detection of B1859+03 is shown for reference.
    }
    \label{fig:dm_time}
\end{figure}

The Arecibo telescope logging data is reported locally which provide pointing,
frequency tuning, and receiver information at approximately one second
resolution. Logs reported the telescope was pointed at a fixed Dec
(+15:11:28.34) and drifting in RA (event detected at RA=14:42:26.18), i.e. a
fixed (alt, az) pointing during the event. No known pulsar or RRAT with a
similar DM is nearby at this pointing.

We considered that the observing band could have been changed in that time.  We
have set up the automated system to restart observations when the \gls{if}
frequency is changed.  During the time of the event, there was no change in the
\gls{if} during that time.

Logs provide the first indication that this event is due to a local source.
Beyond the pointing and \gls{if}, the SCRAM logs report the position of the
receiver turret and \gls{if} report whether \gls{alfa} is active. \gls{alfa} is
at a position angle of approximately $26.64^{\circ}$ in the turret, the system
reported the turret was at $206^{\circ}$. \gls{alfa} is not in, or even near the
focus.  Our commensal observation script checks if \gls{alfa} is active before
we run ALFABURST. This is a check on whether the analogue receiver chain is
properly setup for \gls{alfa}, which almost always means that \gls{alfa} is in
the focus.  But, as we have found out, there are times when this is not true.

The logs do not report the active project or observing schedule. During the time
of the event it appears that no receiver was in use, otherwise, \gls{alfa} would
not have been active, and we see that \gls{alfa} was deactivated approximately
20 minutes after the event when a new observation began.  Looking at the
observation schedule for the morning of December 4, project
P3080\footnote{http://www.naic.edu/vscience/schedule/tpfiles/MichillitagP3080tp.pdf}
was using \gls{alfa} to perform an \gls{frb} survey of the Virgo cluster until
09:00 local time.  After 09:00 local time Project
R3037\footnote{http://www.naic.edu/vscience/schedule/tpfiles/TaylortagR3037tp.pdf}
was scheduled, this is an S-Band RADAR observation.

The average bandpass of Beam 0 and Beam 5 during the time of the event, shows
that the shape and system noise appear different to what is expected during
typical observations (Figure \ref{fig:bandpass_response}).  The detection band
of \gls{alfa} was chosen because it is the most sensitive region of the band,
and relatively flat. But during the event, there is a noticeable shape and slant
to the bandpass which is different than the typical bandpass.  Beam 0 and 5
bandpasses appear similar and have overlapping narrow-band RFI features that
occur during the recorded time window.  The system noise appears to be higher
during the event, which leads to smoother bandpasses than what is typical.  In
the detection pipeline the data is normalized, which removes all absolute
scaling in the process. This indicates that the \gls{sefd} is too low in our
flux calibration. This increase in system noise is due to the change in turret
position, and the \gls{alfa} feed picking up reflections from other equipment in
the dome, and the dome as a warm source.

% watermark:terrestrial-frb-letter/notebooks/ALFABURST_events.ipynb
\begin{figure}
    \includegraphics[width=1.0\linewidth]{figures/bandpass_response.pdf}
    \caption{Average bandpass response during the December 4, 2016 event for
    Beam 0 (green) and Beam 5 (blue). A typical bandpass (red) is plotted for
    reference. These bandpasses have been normalized in the detection pipeline.
    }
    \label{fig:bandpass_response}
\end{figure}

We then looked at previously recorded events close in time to the event.
Approximately 80 seconds earlier another window was recorded which has large
structures across the band (Figure \ref{fig:beam0_dynamic_spec_80s}). Though not
as narrow in time as the event, they appear related to the same phenomenon.

The DM$-$time plot (Figure \ref{fig:beam0_dmtrials_80s}) shows that much of the
structure would be detected as dispersed pulses.  In particular, the structure
around 4 seconds would be detected as a wide-in-time, highly dispersed pulse.
And the structure immediately proceeding it would be detected as a negatively
dispersed pulse.  We do not detect these as pulses because we have limited our
search space to narrow-in-time width pulses and positive \glspl{dm}. Our choice
of search space is reasonable for the type of events we wish to detect given
limited computing resources. But, there are practical advantages to searching
negative \glspl{dm} and wider-in-time pulse widths. We expect astrophysical
pulses to have positive \glspl{dm}. But, system variation and \gls{rfi} can
produce signals which result in positive and negative \gls{dm} detections.  A
statistical measure can be computed in any time window to differentiate times of
significant, but low-level \gls{rfi} or system variation from actual
astrophysical pulses. Testing out to larger pulse widths is computationally
cheap as the time window is decimated, reducing the memory usage. A full
sampling of larger pulse widths or negative \gls{dm} trials similar to the
positive \gls{dm} trials is not necessary. Only a subset can be computed to
provide useful information on the stability of the system and the \gls{rfi}
environment.

\begin{figure*}
    \centering
    % watermark:terrestrial-frb-letter/notebooks/ALFABURST_events.ipynb
    \begin{subfigure}[t]{1.0\textwidth}
        \centering\captionsetup{width=.95\linewidth}
        \includegraphics[width=1.0\textwidth]{figures/D20161204_spect_buf21_Beam0.pdf}
        \caption{Dynamic spectrum shows frequency evolution of the bandpass as a
        function of time with structures similar to the D20161204 event.
        }
        \label{fig:beam0_dynamic_spec_80s}
    \end{subfigure}
    % watermark:terrestrial-frb-letter/notebooks/ALFABURST_events.ipynb
    \begin{subfigure}[t]{1.0\textwidth}
        \centering\captionsetup{width=.95\linewidth}
        \includegraphics[width=1.0\textwidth]{figures/D20161204_dmtrials_buf21_Beam0.pdf}
        \caption{DM trials from -1600 to 1600 show that there would be both
        positive and negative pulse detections during this time window.
        }
        \label{fig:beam0_dmtrials_80s}
    \end{subfigure}
    \caption{Dynamic spectrum (top) and DM-time plot (bottom) of 6 seconds from
    beam 0 approximately 80 seconds before the D20161204 event.
    }
    \label{fig:beamo0_80s}
\end{figure*}

The narrow-in-frequency, periodic RFI at 1468, 1480, 1496, 1504~MHz is not
usually seen in this band. In the high time and frequency resolution view these
short pulses in 12.5~MHz steps have a characteristic dampened harmonic
oscillation due to frequency locking with a \gls{pll} (Figure
\ref{fig:pll_spectrum}). With help from the Arecibo Observatory staff, this has
been identified as instrumentational \gls{rfi} from an instrument that is
routinely used to monitor the local \gls{rfi} environment.

% watermark:terrestrial-frb-letter/notebooks/ALFABURST_events.ipynb
\begin{figure}
    \includegraphics[width=1.0\linewidth]{figures/pll_spectrum.pdf}
    \caption{Dynamic spectrum when a local oscillator in the receiver dome is
    being locked with a phase-locked loop circuit. This LO is not related to
    the receiver analogue mixing chain, but rather it is associated with RFI
    monitoring equipment.
    }
    \label{fig:pll_spectrum}
\end{figure}

The cause of this FRB-like event was likely due to a component of the analogue
chain saturating and entering a non-linear state.  An \gls{rfi} monitoring
program was run on an instrument not included in the telescope logs. This
instrument introduced the narrowband \gls{rfi} due to changing of a \gls{lo}.
The \gls{alfa} receiver was out of the main focus, likely picking up reflections
from the dome, leading to an increase in the system temperature.  This increase
caused some part of the analogue chain, likely an amplifier, to operate beyond
it's linear response region and produce broadband oscillations. With high
certainty, we can classify this as a local event which created the appearance of
an \gls{frb}.  In isolation, and taking into account the one-off, transient
nature of FRBs, the initial Beam 0 detection reasonablely appear to be
astrophysical. It is only with an extended study of the meta-data,
earlier-in-time evolution of the band, and use of multiple beams to confirm that
this is not an FRB of astronomical origin.

\subsection{Low-S/N False-Positive Detections}
\label{sec:low_snr}

As the automated search pipeline is set up to do an extensive search in \gls{dm}
trails, pulse width, and starting time it is reasonable to expect a large number
of low-\gls{snr} false-detections.  \gls{frb} surveys tend to have a significant
minimum \gls{snr} cut-off (6$-$10) due to the large number of false-positive
detections at low \glspl{snr}.  In the ALFABURST and ARTEMIS
\citep{2015MNRAS.452.1254K} surveys the minimum \gls{snr} cut-off is set to 10.

Low-\gls{snr} detections are regularly detected even with high minimum \gls{snr} cut-off.
For example, on July 30, 2015 an apparently 10-sigma event with a DM of
1370~pc~cm$^{-3}$ was detected (Figure \ref{fig:D20150730}). The pulse can
barely be seen in the spectrogram, but in the DM-space plot there is a compact
peak centred at a DM of 1370~pc~cm$^{-3}$. This event is suspicious because the
D20161204 event discussed earlier only has a slightly higher \gls{snr} but is
visible in the spectrogram. The resolution is that the real \gls{snr} of this
event is approximately 6, while a the reported value of 10 is based on a longer
running average computed in the automated pipeline. Locally in time there is an
increase in system noise, leading to a reduced \gls{snr}. The increase in system noise
is caused by the receiver turret rotating while the ALFA receiver was still
active.

% watermark:terrestrial-frb-letter/notebooks/ALFABURST_events.ipynb
\begin{figure}
    \includegraphics[width=1.0\linewidth]{figures/D20150730_buf23_Beam6_dmtrial.pdf}
    \caption{DM-space plot and time series of a high DM event with reported
    \gls{snr} above 10-sigma on July 30, 2015. After re-analysis and review of
    the telescope meta-data it was determined that this event was due to local
    system noise.
    }
    \label{fig:D20150730}
\end{figure}

The reported detection of a pulse from FRB121102 with APERTIF \citep{atel10693}
had an \gls{snr}$\approx 4$. In a blind survey across different sky positions
and DM trials this would not be a significant detection. But, the sky position
and \gls{snr}-maximized DM of FRB121102 is known, thus a lower \gls{snr}
detection may be reasonable to report.  We require a large \gls{snr} to validate
a detection.  As the parameter space (DM trials, pulse width range) and number
of observations grow we expect to see an increase in the number of
high-\gls{snr}, false-poisitive events.

\subsection{Detected RADAR Event with ARTEMIS}
\label{sec:LOFAR_RADAR}

ARTEMIS \citep{2015MNRAS.452.1254K} is  an \gls{frb} search survey, similar to
ALFABURST, run on the LOFAR-UK station at Chilbolton Observatory. The ARTEMIS
survey uses a similar fractional bandwidth ($\sim 0.04$) to ALFABURST, but is
centred at 145~MHz. In this survey known pulsars regularly transit the beam that
are fixed in local coordinates, and single pulses are routinely observed.
Occasionally, \gls{rfi} events are detected by the automated pipeline.  A
particularly interesting event for which the \gls{snr} was maximized at a
\gls{dm}$= 85$~pc~cm$^{-3}$ is shown in Figure \ref{fig:lofar_dynamic}.  The
narrow-in-time pulse can be seen in the dynamic spectrum at frequencies above
146 MHz, but not at lower frequencies where it could be hidden by narrow band
\gls{rfi}.  The dedispered time series shows a high-\gls{snr} detection of a
pulse of approximately 20~ms in width.

% watermark:terrestrial-frb-letter/notebooks/LOFAR_RADAR.ipynb
\begin{figure}
    \includegraphics[width=1.0\linewidth]{figures/LOFAR_dynamic.pdf}
    \caption{A dispersed pulse detected by the automated ARTEMIS search
    pipeline. The dispersed pulse only appears over a narrow band of just over
    1~MHz.
    }
    \label{fig:lofar_dynamic}
\end{figure}

The beam pointing during the time of the event is not associated with any known
pulsar or RRAT around a DM$\sim$85~pc~cm$^{-3}$. This DM is low compared to
reported \gls{frb} detections.
% TODO: pointing, SNR, maximized DM and time decimation, NE2001 model for
% distance
Plotting the event in DM-space across the ARTEMIS DM range
($0-320$~pc~cm$^{-3}$, Figure \ref{fig:lofar_dm_time}) shows a strong, compact
detection as expected of a dispersed pulse during a time of minimal \gls{rfi}.

% watermark:terrestrial-frb-letter/notebooks/LOFAR_RADAR.ipynb
\begin{figure}
    \includegraphics[width=1.0\linewidth]{figures/LOFAR_dm_time.pdf}
    \caption{DM-space plot of the event shows a strong, compact detection around
    a DM of 85~pc~cm$^{-3}$ with no other apparent events during at the time.
    }
    \label{fig:lofar_dm_time}
\end{figure}

Since the pulse is apparent in only part of the band it's origin is suspicious.
It could be that the pulse is broadband and only the brightest component is seen
in the 146-147~MHz region. Or, it could be a band-limited pulse.  The ARTEMIS
search pipeline, like ALFABURST and many other search pipelines, decimates the
dynamic spectrum in time to search over a range of pulse widths.  The \gls{snr}
of the event shown in Figure \ref{fig:lofar_dynamic} is maximized for a time
decimation factor of 64. With a native resolution of $327.68~\mu s$, this
results in a decimated time resolution of 20~ms. Plotting the event at a higher
time resolution (Figure \ref{fig:lofar_dynamic_high}) we see a distinct
repeating pattern in time previously not visible. This is a linearly frequency
modulated signal used for pulse compression in RADAR.

% Linear Frequency Modulated Pulse Waveforms
% https://en.wikipedia.org/wiki/Pulse_compression
% http://www.radartutorial.eu/02.basics/Stepped%20Chirp%20Radar.en.html
% google: radar chirp l band, radar chirp quadratic, stepped chirp l band
% stepped chirp waveform
In RADAR observations, the bandwidth of the transmitter provides information on
the range and direction of a target. A narrow band RADAR transmitter can be used
to approximate a larger bandwidth by modulating the frequency of the transmitted
pulse. The narrow band (in frequency) pulse is stepped in frequency across a
transmission band. Between each step the pulse is not transmitted, resulting in
the gaps in time between pulses in Figure \ref{fig:lofar_dynamic_high}. Linear
frequency modulation is the most typical form of chirp compression, but
non-linear methods are also used. A windowing function can be applied to the
pulse which can produce quadratic frequency modulations.  There are a number of
allocated usages in this band which could be the source of the observed RADAR
pulse \citep{ofcom2017}. RADAR is also used from UHF to C-band, covering a wide
range of frequencies at which \gls{frb} surveys operate. As RADAR is used for
commercial and military purposes, most of these signal specifications and
modulations are proprietary.

% watermark:terrestrial-frb-letter/notebooks/LOFAR_RADAR.ipynb
\begin{figure}
    \includegraphics[width=1.0\linewidth]{figures/LOFAR_dynamic_high_res.pdf}
    \caption{A zoomed in view of the event in Figure \ref{fig:lofar_dynamic} at
    high time and frequency resolution shows the distinct pattern of a linearly
    frequency modulated RADAR pulse.
    }
    \label{fig:lofar_dynamic_high}
\end{figure}

As the RADAR signal is a dispersed pulse we expect to detect such signals with
\gls{frb} pipelines.  Verification of this event is straightforward when
examining the high-time and frequency resolution data that reveal the pulsed
nature of the event.  The frequency allocation and the use of standard RADAR
techniques across the radio band are clear indicators of the origin of this
event.  

Verification becomes challenging when the desire is to run the system in
real-time with immediate follow-up with additional telescopes.  Expert knowledge
in the instrument, radio signals and visual acuity were necessary to identify
this rare RADAR event.  Automating this process is difficult, so if we desire an
automated VOevent trigger, we should expect a number of false-positive triggers.

\subsection{XAO Repeating Event}
\label{sec:xao_event}

The 25-m Nanshan Telescope at \gls{xao} is currently running an FRB survey. The
survey covers over 300~MHz at L-band, sampling at $64 \; \mu s$ resolution. On
November 18, 2016 hundreds of bright, dispersed pulses were detected. The pulses
varied in \gls{snr}, but had the same \gls{snr}-maximized DM of 531.8~pc~cm$^{-3}$. The
pulses show a distinct double peak (each $\sim 2$~ms wide) separated by $\sim
3$~ms (Figure \ref{fig:xao_dynamic}). The pulse is only apparent in a about a
quarter of the band.

% watermark:/home/griffin/data/XAO/coadd_pulses.ipynb
\begin{figure}
    \includegraphics[width=1.0\linewidth]{figures/XAO_pulse_dynamic.pdf}
    \caption{Example of a detected pulse with the Nanshan Radio Telescope at XAO
    which is \gls{snr} maximized at a dedispersion of $531.8$~pc~cm$^{-3}$. Hundreds
    of such pulses were detected over a period of a few days. The orange bars
    represent regions that have had significant constant-in-time RFI removed.
    }
    \label{fig:xao_dynamic}
\end{figure}

The pulses were seen at different pointings across the sky. Most were detected
at a low altitude pointing angle, but some were detected close to zenith.  A
periodicity search revealed a periodicity of $\sim 1.7$~s, but the residuals
were significantly higher than that of a typical pulsar periodicity search. This
along with the pointing variability rules out a pulsar, or an astronomical
source in general.

A dispersion relation parameter estimation found that pulses deviate from a
$\nu^{-2}$ relation by $1.5 \sigma$.  Figure \ref{fig:xao_summed} shows the sum
of the dedispersed (using DM 531.8~pc~cm$^{-3}$) phase-aligned pulses.  The
addition produces an S-shaped broadband pulse with frequency structure that is
not seen in the individual pulses.  There is also regular, narrow-band frequency
modulation in the pulse structure.  The structure is due to non-linearly
frequency modulation, used in chirped RADAR systems, which is different from the
linear modulation observed in Section \ref{sec:LOFAR_RADAR}.

% watermark:/home/griffin/data/XAO/coadd_pulses.ipynb
\begin{figure}
    \includegraphics[width=1.0\linewidth]{figures/XAO_summed_dynamic.pdf}
    \caption{The dynamic spectrum of 182 high-\gls{snr} pulses detected with the
    Nanshan Radio Telescope summed together.  The low-level, non-linear
    frequency modulation can be seen across the band. The orange bars represent
    regions that have had significant constant-in-time RFI removed.
    }
    \label{fig:xao_summed}
\end{figure}

A thorough search of possible local sources, such as new equipment, vehicles,
and aircraft was preformed. No obvious candidate was found. Detection of
multiple pulses at different beam pointings indicates the source may be directly
illuminating the feed. The source is likely not due to system electronics
because of the complex frequency modulation but rather due to a local
transmitter. Had only a single pulse been detected with the telescope, for
example if the source was weaker, or the telescope was only sensitive to the
highest \gls{snr} event, it would be difficult to show that the event was due to
RFI. Multiple reported \glspl{frb} do not cover the entire observing band. This
has been explained by various astronomical effects (scintillation, plasma
lenses). In this case, it would be reasonable to report a single pulse as an
\gls{frb}. The Nanshan L-band receiver is a single pixel system. It could be
that a multi-beam system such as the Parkes Multi-beam or ALFA would detection
these events in many of the beams. In that case, the detections could be
classified as local \gls{rfi}.

The non-linear modulation scheme can produce a range of apparent
\gls{snr}-maximized events depending on the observing band compared to the pulse
transmission band (Figure \ref{fig:xao_simulated_dm}). A simple model of the
pulse can be produced by using a logistic function which covers a bandwidth from
$\nu_{\textrm{p,0}} = 1350$~MHz to $\nu_{\textrm{p,1}} = 1750$~MHz (Figure
\ref{fig:xao_simulation_diagram}). The time duration of the pulse $\Delta
t_{\textrm{pulse}}$ is fixed such that central frequency range of the pulse has
an \gls{snr}-maximized DM of approximately 530~pc~cm$^{-3}$. The pulse is given
a width ($\Delta t_{\textrm{width}}$) by convolving the logistic function with a
3~ms wide Gaussian. The pulse is set to unity amplitude across the extent of the
band.

% figures/simulation_digram.odg
\begin{figure}
    \includegraphics[width=1.0\linewidth]{figures/simulation_diagram.pdf}
    \caption{Non-linearly modulated pulse and receiver model used to simulate
    the response and \gls{snr}-maximized DM fit to a pulse similar to the one detected
    at XAO.
    }
    \label{fig:xao_simulation_diagram}
\end{figure}

A receiver model is used to simulate the observation. This model is
parameterized by central observing frequency $\nu_{\textrm{obs,c}}$, bandwidth
$\Delta \nu_{\textrm{obs}}$, number of observing frequency channels
$n_{\textrm{freqs}}$, and per channel noise $\sigma_{\textrm{chan}}$. The
bandpass is modelled as a Gaussian.

% watermark:/home/griffin/data/XAO/Non-Linear-Pulse-Simulation.ipynb
\begin{figure}
    \includegraphics[width=1.0\linewidth]{figures/simulatedRADARdm.pdf}
    \caption{\gls{snr}-maximized DM trials for a simulation of the XAO
    non-linearly modulated pulse over a range of central observing frequencies,
    and receiver bandwidths.
    }
    \label{fig:xao_simulated_dm}
\end{figure}

Detection of the pulse is simulated across a range of central observing
frequencies and receiver bandwidths. The \gls{snr}-maximized DM is determined by
performing a trial DM search from 0 to 4000~pc~cm$^{-3}$ in integer increments.
For all bandwidths, when the central observing frequency is near the centre of
the pulse the apparent DM of $\sim530$~pc~cm$^{-3}$ maximizes the detection
\gls{snr}. However, when the central observing frequency is shifted relative to
the pulse centre frequency a range of \gls{snr}-maximized DM's are reported,
depending on the receiver bandwidth. This non-linearly modulated pulse will
appear very similar to an FRB detection at a receiver-dependent DM trial value.
The top and bottom of the pulse will likely appear as RFI due to the curve of
the logistic function.

% TODO: did this happen?
%The source was only detected over the course of a day, and the origin remains
%unknown.  Observations a XAO continue, with an interest in observing in November
%2017 to see if the signal returns on an annual schedule.


\section{Verification Criteria}
\label{sec:verify_crit}

Given that systematic effects and \gls{rfi} can produce apparent \glspl{frb} it
is useful to look back at previously reported \glspl{frb} that don't fit the
ideal broad-band, single component structure that detection pipelines are
optimized for.

% every FRB paper starts out with a definition, make that the fundamental
% checklist. which FRBs fail or succeed?
% TODO: what are questions that can be answered? aristotlian catagorization
% Check list of items:
%   broadband? scattered? multiple components?
%   multi-beam system
%   variation from the average bandpass
%   pointing? near the horizon?
%   repeats? repeats at different pointings?
%   resolved in time and frequency?
%   dispersion index fit
% Table of reported FRBs and events from this paper

% TODO: references
\begin{table*}
\centering
\begin{tabular}{ r l l r r }
Name      & Telescope & DM (pc~cm$^{-3}$)& Width (ms) & \gls{snr}  \\
\hline
FRB010125 & Parkes  & $790 \pm 3$        &  9.4 	&	17   \\ 
FRB010621 & Parkes  & $745 \pm 10$       &  7		&	16.3 \\
FRB010724 & Parkes  & $375$              &  5		&	23   \\ 
FRB090625 & Parkes  & $899.55 \pm 0.01$  &  1.92	&	30   \\ 
FRB110220 & Parkes  & $944.38 \pm 0.05$  &  5.6		&	49   \\ 
FRB110523 & GBT     & $623.3 \pm 0.06$   &  1.73	&	42   \\ 
FRB110626 & Parkes  & $723 \pm 0.3$      &  1.4		&	11   \\ 
FRB110703 & Parkes  & $1103.6 \pm 0.7$   &  4.3		&	16   \\ 
FRB120127 & Parkes  & $553.3 \pm 0.3$    &  1.1		&	11   \\ 
FRB121002 & Parkes  & $1629.18 \pm 0.02$ &  5.44	&	16   \\ 
FRB121102 & Arecibo & $557 \pm 2$        &  3		&	14   \\ 
FRB130626 & Parkes  & $952.4 \pm 0.1$    &  1.98	&	21   \\ 
FRB130628 & Parkes  & $469.88 \pm 0.01$  &  0.64	&	29   \\ 
FRB130729 & Parkes  & $861 \pm 2$        &  15.61	&	14   \\ 
FRB131104 & Parkes  & $779 \pm 1$        &  2.08	&	30   \\ 
FRB140514 & Parkes  & $562.7 \pm 0.6$    &  2.8		&	16   \\ 
FRB150215 & Parkes  & $1105.6 \pm 0.8$   &  2.88	&	19   \\ 
FRB150418 & Parkes  & $776.2 \pm 0.5$    &  0.8		&	39   \\ 
FRB150610 & Parkes  & $1593.9 \pm 0.6$   &  2		&	18   \\ 
FRB150807 & Parkes  & $266.5 \pm 0.1$    &  0.35	&	     \\ 
FRB151206 & Parkes  & $1909.8 \pm 0.6$   &  3		&	10   \\ 
FRB151230 & Parkes  & $960.4 \pm 0.5$    &  4.4		&	17   \\ 
FRB160102 & Parkes  & $2596.1 \pm 0.3$   &  3.4		&	16   \\ 
FRB160317 & UTMOST  & $1165 \pm 11$      &  21		&	13   \\ 
FRB160410 & UTMOST  & $278 \pm 3$        &  4		&	13   \\ 
FRB160608 & UTMOST  & $682 \pm 7$        &  9		&	12   \\ 
FRB170107 & ASKAP   & $609.5 \pm 0.5$    &  2.6		&	16   \\ 
FRB170827 & UTMOST  & $176.4$            &  0.4		&	90   \\ 
FRB170922 & UTMOST  & $1111$             &  26		&	22   \\
\hline
B1859+03  & Arecibo & $402.080$          &  11      &   20   \\ % high-DM pulsar
D20161204 & Arecibo & $293$              &  20      &   10   \\
D20150730 & Arecibo & $1370$             &  0.5     &    6   \\
ARTEMIS RADAR & LOFAR-UK & $85$          &  20      &        \\
XAO Repeater & XAO  & $531.8$            &  2       &   10>  \\
Perytons  & Parkes  & $\sim400$          &  18.5    &   10>  \\
\end{tabular}
\caption{Previously reported FRBs and terrestrial events discussed in this work.}
\label{tbl:frbs}
\end{table*}

\subsection{FRB130729}

\cite{2016MNRAS.460L..30C} reported detection of FRB130729 along with four other
\glspl{frb} from the HTRU survey. They note that flux only appears in the lower
half of the band, has a double peak structure, and is potentially due to
terrestrial \gls{rfi}.  They report a \gls{dm} of 861~pc~cm$^{-3}$ for
FRB130729, this maximizes the detection \gls{snr}, but, by using a DM of
852~pc~cm$^{-3}$ (Figure \ref{fig:FRB130729}) the double-peak structure can be
seen more evident. By using this lower \gls{dm} the pulse width of the two
components can be seen to be much more narrow than the reported $\sim 16$~ms.
The two components are separated by approximately 10~ms, with the second
component appearing only about half as wide in bandwidth compared to the first. 

% FRB130729
% aslxlap07:/local/griffin/data/FRB/FRB130729/FRB130729.ipynb
\begin{figure}
    \includegraphics[width=1.0\linewidth]{figures/FRB130729.pdf}
    \caption{FRB130729 dedispersed with a DM of 852~pc~cm$^{-3}$, this is
    different from the \gls{snr}-maximized DM of 861~pc~cm$^{-3}$. Using this DM shows
    that the detected FRB has two components distinct components separated by
    approximately 10~ms. Data is presented at the native recorded resolution of
    64~$\mu$s, 390~kHz convolved with a Gaussian smoothing filter of size
    512~$\mu$s, 3.125~MHz.
    }
    \label{fig:FRB130729}
\end{figure}

\subsection{FRB140514}

Detection of FRB140514 was reported in \citep{2015MNRAS.447..246P} and was
reported to have significant circular polarization. But, the flux is primarily
concentrated between 1240~MHz and 1270~MHz. This small fractional bandwidth is
not too different from the ARTEMIS event we have reported earlier.  If this
component is removed, for example if the observation was at a slightly different
frequency, or smaller bandwidth, the detected \gls{snr} drops below 10.
Further, many anthropogenic radio transmitters are circularly polarized. But,
it could also be that there is a plasma lensing effect
\citep{2017ApJ...842...35C} which has been used to explain the variable spectral
index and bandwidth of FRB121102, and FRB140514 is indeed astrophysical.

% FRB140514
% aslxlap07:/local/griffin/data/FRB/FRB140514/FRB140514.ipynb
\begin{figure}
    \includegraphics[width=1.0\linewidth]{figures/FRB140514.pdf}
    \caption{The flux of FRB140514 is concentrated in a few, narrow fractional
    bandwidth regions, primarily centred around 1260~MHz.  Data is presented at
    the native recorded resolution of 64~$\mu$s, 390~kHz convolved with a
    Gaussian smoothing filter of size 512~$\mu$s, 3.125~MHz.
    }
    \label{fig:FRB140514}
\end{figure}

\subsection{False-Positive Reporting}

During a search for anomalous signals it should be expected that there are a
number of events reported which are false positives. Such as Perytons
\citep{2011ApJ...727...18B} and the events reported in earlier sections. Events
such as FRB130729 and FRB140514 are on the edge of verifiability. They do not
fit the classic \gls{frb} model but given the available observing and system
information, and lack of an external, anthropogenic explanation they are hard
to discount as not being of astrophysical origin. It is necessary to report
all of these events and to provide as much evidence as possible to explain the
origin of such events.

\section{Detection Reporting and Future Observational Methods}

% TODO: decision tree
% TODO: reference varying shapes of FRB121102

The one-off nature of \glspl{frb} makes it essential that when reporting on a
detection or triggering a follow-up observation that significant due-diligence
is done in order to verify a true-positive detection as much as possible. Over the
past decade of \gls{frb} surveys a number of techniques have been developed to
efficiently filter for dispersed pulses. The vast majority of events are flagged
by \gls{rfi} excisers and setting a sufficiently high minimum \gls{snr}
threshold. This has the effect of creating potential false-negative events (i.e.
FRBs classified as RFI), but is not useful with false-positive detections. The
mystery and rarity of FRBs makes it difficult to differentiate astrophysical FRBs
from local sources based on the dynamic spectrum alone. Thus, additional
information on the observing system should be used to provide robustness to the
detection.

\subsection{Minimum Reporting}

At a minimum, a reasonable detection should be reported with the observed data
made publicly available.  This allows independent verification, and can be used
as input data sets to test pipeline development.  This is different from other
transient detections in that astronomical pointing and observing frequency is
sufficient to do a follow-up observation. If \glspl{frb} are indeed one-off
event, then the detection data is the only data that will ever be available. A
minimum reporting should include:

\begin{enumerate}
    \item Filterbank which fully encompasses the extent of the dispersed pulse.
    \item Time and frequency resolutions, DM, and start time of detection.
    \item Dedispered time series.
    \item Astronomical pointing, observing frequency, and other telescope
    observing parameters.
    \item For a multi-beam system, filterbanks of each beam covering the extent
    of the pulse.
    \item A list (or guide) of software and parameters used to generate
    detections and plots.
\end{enumerate}

These requirements are typically included in past reported \gls{frb} detections.
The last point, a guide on which software was used is often not reported.
Different software and data formats can have slightly different results, such as
the reported \gls{snr} or time definition.  FRBCAT \citep{2016PASA...33...45P}
provides a well-curated repository for this information. The atypical
\glspl{frb} shown in Section \ref{sec:previous_frbs} are from the online data
repositories linked in FRBCAT. Most reported \glspl{frb} detected with Parkes
have publicly accessible data repositories which meet the reporting requirements
stated above. As of this writing, there is no publicly available data for
approximately a quarter of the reported \glspl{frb}. Upon publication of a
detection, this data should be made available.
% FRB110703: dodgy filterbank
% FRB110523, FRB121102, FRB150807, FRB160317, FRB160410, FRB160608, FRB170107,
% FBR170827
% FRB150807: bright, maybe RFI

\subsection{Desirable Reporting}

As has been shown in the previous section there are instrumentational and RFI
sources which can produce the appearance of \glspl{frb}. Minimum reporting is
insufficient for these non-astrophysical detections. These detection were only
excluded with additional understanding of the telescope and observing status.
Primarily this would be to expand the reported data in time before and after the
detected event. Desirable reporting would include:

\begin{enumerate}
    \item Filterbank data which not only encompasses the detected pulse, but
    includes data over a longer time span, e.g. a few minutes before and after
    the detection. If multiple feeds were in use, data from all feeds should be
    made available.
    \item An expected bandpass model, and bandpass model during detection. And,
    the measured gain variation over the observation.
    \item Telescope observing parameters and pointing over the observation,
    including the local (altitude, azimuth) pointing.
    \item A DM-space plot of the detection, along with a negative DM-space
    sampling.
    \item A dispersion fit to the pulse.
    \item Raw voltage or complex spectral data before power detection and
    integration.
\end{enumerate}

Reporting of the bandpass and gain variations, along with a long time data set
allows the state of the telescope to be understood. A deviation from normal
operation, e.g. analogue electronics going into non-linear states, has the
potential to produce spurious events which on a small time-scale can look
FRB-like.

A DM-space plot provides a useful diagnostic to protect against observation
bias. An ideal detection, such as in Figure \ref{fig:dm_time_B1859}, should
appear compact in this space, with no other detections at different trial DMs.

\subsubsection{Multi-site Observations}

Clear evidence for an FRB to be of astrophysical origin is the simultaneous
detection of signal with telescopes at multiple sites. Multiple detectors, such
as in LIGO, are essential for false-positive rejection. Coordinating telescope
observations in logistically difficult, for example many FRB surveys run
commensally during targetted observations, but would prove valuable in reporting
detections.  Further, detection of an FRB at multiple bands would provide
insight into the bandwidth characteristics of the event.

Interferometric arrays provide a similar advantage to multi-site observations.
Detections with individual recievers indicate an event is not due to systematics
such as those in Section \ref{sec:D20161204}. And, the array can be used to
localize the sky position. A detection could still be due to an RFI source local to
the array, though the physical seperation of the elements and interferometric
techniques could be used to determine if the source is in the near field.

\subsubsection{Low-\gls{snr} Follow-up Search}

Once a potential detection has been made at a given minimum \gls{snr} threshold,
a second search should be performed focused on a DM range centred on the
\gls{snr}-maximzed DM. This search should go to a lower minimum \gls{snr} threshold.
Preferably this search would be over the entire observation. This test will
serve to show whether there are additional events, similar to the XAO events,
just below the minimum threshold, or this is a true outlier event.

\subsubsection{Negative DM Sampling}

Current surveys typically search out to extreme \glspl{dm}, in the case of
ALFABURST we search out to a \gls{dm} of 10000~pc~cm$^{-3}$ as the additional
computational cost is minimum. If the \gls{dm}-redshift relation is
approximately correct, searching to such high \glspl{dm} is sufficient to sample
out to the very early universe. We expect an astrophysical source to be
dispersed by a positive \gls{dm} value. As such we do not search for negatively
dispersed pulses. But, \gls{rfi} does show up as both positive and negative
dedispersion detections (Figure \ref{fig:beamo0_80s}). As the dedispersion
search is no longer computationally limited it is possible to search the
negative \gls{dm} space at low additional cost. This would be useful as a
statistical statement about the number of positive versus negative detections to
quantify the amount of \gls{rfi} during a detection. The full negative \gls{dm}
space would not need to be searched, a regular sampling of the space would be
sufficient.

\subsubsection{Dispersion Fitting}

FRB surveys, by design, search for broadband signals which follow a $\nu^{-2}$
cold plasma dispersion relation. A distant astrophysical pulse should follow
this relation, and any deviation from this relation is strong evidence for the
signal being artifically generated, e.g. the XAO repeating event in Section
\ref{sec:xao_event}. This dispersion relation can be tested on the dynamic
spectrum of a potential FRB.

The incoherent dedispersion operation aligns frequency subbands based on such a
frequency-dependent time delay. The dynamic spectrum is averaged in frequency to
produce a time series. Peaks above a threshold S/N are recorded as possible
detections. But, such a detection is not necessarily maximized in S/N with at
the cold plasma dispersion index ($\alpha = -2$). The frequency-dependent delay
of a dispersed pulse can be modelled as
%
\begin{equation}
\Delta t = A \, (\nu_1^{\alpha} - \nu_2^{\alpha}) + \Delta t_0,
\end{equation}
%
where $A$ is the scaling amplitude, for $\alpha = -2$, $A = {\rm DM}$, $\nu_1$
is the reference frequency (usually taken to be the highest frequency subband),
and $\nu_2$ is the observing frequency of a subband.

To perform the fit, the pulse signal needs to be seperated from the noise. This
can be done with various methods. A simple method is to pick the peak flux value
of each subband, that position corresponds to a delay and the flux value is used
as a weight. A more advanced method would be to find the position peak after
correlating the subband with a Gaussian pulse with a width similar to the
detected pulse. In any method, it is useful to exclude subbands where the pulse
is at a low S/N, and limit the maximum delay region to near the predicted
$\nu^{-2}$ delay. The initial parameters for the fit are set to be the original
$\nu^{-2}$ detection parameters.

To produce a reasonable fit and error estimate of the dispersion index
sufficient power needs to be present across a large enough fractional bandwidth.
For example, system such as ALFABURST and ARTEMIS have narrow fractional
bandwidth ($\sim 0.05$) which is insufficient to differentiate between a linear
chirp ($\alpha = -1$) and a cold-plasma dispersed pulse. Similary, this is the
case for a potential FRB that is only detected at significant S/N in a small
fractional bandwidth (e.g. FRB140514).

If the complex voltages are captured during a detection a better dispersion
relation fit can be performed by doing coherent dedispersion. It is often not
practical to store all the unaccumulated voltages during a survey. But, either
detection triggers or short-term storage of the complex voltages be useful for
this test which provides strong evidence for the astrophysical nature of a
pulse.

\subsection{Future Observational Methods}

Beyond detection of more \glspl{frb}, the goals of current surveys are to
localize the source to host galaxies, and detect pulses across broader
bandwidths across the radio spectrum.

Localization requires the use of interferometric arrays. In the case of a
compact array, a high-time resolution correlation matrix is recorded. And, in
the case of \gls{vlbi} the raw complex voltages are recorded, which can be used
to perform coherent dedispersion resulting in a higher resolution (time and
frequency) dynamic spectrum of the pulses.  Related, multi-site coordinated
observations would remove most, if not all, instrumentational and RFI sources of
false-positive detections.

Detection of \glspl{frb} at multiple frequencies not only adds to the scientific
understanding of the sources, but also help to verify that they are
astrophysical.  Due to historical development of receivers for pulsar searches,
most \glspl{frb} have been detected at L-band frequencies. Though, FRB110523
detected with the GBT and the multiple \glspl{frb} detected with UTMOST occurred
at UHF frequencies.  Only FRB121102 has been detected above L-band.  Pulses from
FRB121102 have been detected across a 4 GHz band (4 - 8~GHz) \citep{atel10675}.
Such wide bands show the pulse structure goes beyond the bandwidth of known
source of \gls{rfi} (e.g. modulated RADAR).

\section{Conclusion}

%narrative: here are some odd FRB like things, here are some odd reported
%FRBs, it is hard to know if they are real and a ~10% false-positive rate is
%probably fine, but bursts need to be fully reported

% TODO: follow text
%This creates a challenge of automating the triggering of follow-up with other
%telescopes. Either there will be an excess of false-positive triggers but with a
%short delay between detection and triggering.  Or, a non-automated, expert
%examination of the event is required to verify, creating a delay in any
%follow-up. Automated follow-up should be triggered when there is high confidence
%in the likelihood of a true-positive.

\glspl{frb} are unique astronomical sources from an observational point of view
as so far no follow-up observations have been able to verify a source using a
different telescope or observing frequency, except in the case of FRB121102
which is known to repeat. Thus, it is necessary to provide reasonable evidence
for an astrophysical origin when reporting a detection. 

As the size and number of \gls{frb} surveys continues to increase, one expects
an increase in the number of false-positive detections, even as rejection models
are improved. There is a narrow line between true, astrophysical \glspl{frb} and
false-positive events. Ancillary evidence of the telescope state and observation
provide robust evidence for a true-positive detection. 

Reporting of false-positive events, even if the source is not explained, helps
to improve the robustness of search pipelines against systematics and \gls{rfi}.
Reporting these events also help improve the case for \glspl{frb} being of
astrophysical origin, just as the explanation for Perytons
\citep{2015MNRAS.451.3933P} removed doubt about detections using Parkes.

Though many previously reported \gls{frb} detections have provided sufficient
data and telescope information, more can be provided to present a stronger case
for the astrophysical nature of the detection. Any reported detection at the
minimum should make the observation publicly accessible. Further, as
observing with any radio telescope requires high expert knowledge, it is useful
for the expert observer to include a statement of the telescope status at the
time of a detection.

Current and future surveys which search across large fractional bandwidths, and
localize with interferometric observations will provide further evidence to
associate an \gls{frb} detection with an astrophysical source.

Jupyter notebooks and the filterbanks files are hosted on our
public git
repository\footnote{https://github.com/griffinfoster/terrestrial-frb-letter}.

\bibliographystyle{mnras}
\bibliography{frb-detections} 

% Don't change these lines
\bsp	% typesetting comment
\label{lastpage}
\end{document}

% End of mnras_template.tex
